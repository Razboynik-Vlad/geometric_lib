\documentclass{article}
\usepackage[utf8]{inputenc}
\usepackage[russian]{babel}
\usepackage{amsmath}
\usepackage{graphicx}
\usepackage{listings}
\usepackage[colorlinks=true, urlcolor=blue, linkcolor=black, urlbordercolor={1 1 1}]{hyperref}
\usepackage[dvipsnames]{xcolor}


\lstloadlanguages{Python}
\lstset{
    language=TeX,
    basicstyle=\ttfamily,
    keywordstyle=\color{Blue},
    emph={area,perimeter},
    emphstyle=\color{Orange},
    numbers=left,
    numberstyle=\tiny,
    stepnumber=1,
    numbersep=5pt,
    tabsize=2,
    mathescape=true
}
\title{Работа с LaTeX. Лабораторная работа № 3}
\author{Владислав Хозеев М3110, Сенсей:  Жуйков Артём Сергеевич}
\date{\today}

\begin{document}

\maketitle

\tableofcontents
\newpage
\section{Введение}
    \paragraph{} В данном документе будет рассмотрена библиотека работа с площадью и периметром на примере квадрата и круга в реализации на языке программирования Python.
    

\section{Примеры}


\subsubsection{Квадрат}

\begin{lstlisting}[language=Python]
def area(a):
    return a * a

def perimeter(a):
    return 4 * a
\end{lstlisting}

Эта программа вычисляет площадь и периметр квадрата используя заданную сторону квадрата.

\begin{equation}
A = a^2
\end{equation}

\begin{equation}
P = 4a
\end{equation}

где $A$ -- это площадь, $P$ -- это периметр, и $a$ -- длина стороны квадрата.


\subsubsection{Круг}

\begin{lstlisting}[language=Python]
import math

def area(r):
    return $\pi$ * r * r

def perimeter(r):
    return 2 * $\pi$  * r
\end{lstlisting}

Программа вычисляет площадь и периметр круга.

\begin{equation}
A = \pi r^2
\end{equation}

\begin{equation}
P = 2\pi r
\end{equation}

Где $A$ -- это площадь, $P$ -- это периметр, and $r$ -- радиус.

\section{Cсылка на GitHub, для оценки исходного кода}

\href{https://github.com/Razboynik-Vlad/geometric_lib}{Cсылка на GitHub}

\end{document}